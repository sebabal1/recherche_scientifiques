\documentclass[10pt,a4paper]{article}
\usepackage[utf8]{inputenc}
\usepackage{amsmath}
\usepackage{amsfonts}
\usepackage{amssymb}
\begin{document}

\section{Machine Learning}

Le Machine Learning est une sciences moderne qui permet d'effectuer des prédictions à partir de données qui se basent sur des statistiques. \\

Cette méthode fait partie d'une branche de l'intelligence artificielle qui englobe plusieurs méthodes afin de créer automatiquement des modèles à partir de données. \\

Le Machine Learning prend de l'expérience au fur et à mesure que l'algoithme est exposé à davantages de données. C'est ainsi que celui ci permet de s'améliorer. Un programme informatique quand à lui, il suit que des instructions précises.\\

Le Machine Learning peut etre différencié par deux types d'algorithmes : supervisés et non supervisés.L'apprentissage supervisé utilise des données qui sont déjà connues par le model avec une étiquette. A la fin de son entrainement,le model pourra être capable de retrouver des données dans le même domaine dont les données n'ont pas d'étiquette.\\

L'apprentissage non supervisé, quand à lui, permet d'entrainer le modèle sans étiquette. La machine cherche parmis les données sans indices et permet de découvrir les tendances.L'apprentissage par renforcement, ce modèle permet d'entrainer la machine avec un objectif bien précis.Le modèle a un système d'échecs et d'erreurs. \\


\subsection{En quoi consiste le Machine Learning}

Le Machine Learning est présent partout sur la toile, cela va au moteur de recherche comme Google. Nos applications tels que Siri et Alexa. Les fils d'actualités comme Facebook et Twitter. Toutes ces platformes stockent des données sur les utilisateurs afin de comprendre et d'améliorer leurs performances. Ces données serviront à mieux cibler ce que les utilisateurs aiment. La machine pourra ainsi proposer plus facilement des recommandations ou des résultats pour des recherches.


\subsection{Big Data et Machine Learning}

Avec un grand nombre de données les outils analytiques ne savent pas traiter autant de donnés pour l'exploiter à bon escient. Un volume de données trop large empèche une analyse compréhensive. Les corrélations et les relations entre ces données sont trop importantes ce qui complique la tache pour des analystes. 

C'est pour cette raison que le Machine Learning est idéal pour du Big Data. Cette technologie pourra extraire des valeurs qui proviennent de cette source de données sans passer par l'intermèdiaire d'un être humain.

Sans le Big Data, le Machine learning et l'intelligence artificielle ne sont rien. Les données sont le moyen pour le Machine Learning d'apprendre et de comprendre comment pensent les humains. 

La technologie est capable d'apprendre en allant chercher elle même des ensembles de données pour les analyser et devenir plus précise.

\subsection{Deep Learning, sous domaine du Machine Learning}

Le Deeo Learning est basé sur un système d'un réseau neuronal inspiré des systèmes cérébraux. Ce type d'apprentissage est d'y supervisé car c'est le développeur qui vont décidé sur que type d'apprentissage ils vont lancer le Deep Learning. Cette technique à besoin d'énormément de données (Data Lake).

Un réseau de neuronnes est un ensemble de neuronne connecté entre eux. 

\label{Pour la partie sur le Machine Learning https://www.lebigdata.fr/machine-learning-et-big-data}


\label{https://www.youtube.com/watch?v=QR1SnCRungE&ab_channel=AlainOlivetti} 




\end{document}